\documentclass[11pt]{article}

\begin{document}

\title{CMAC: Cipher-based MAC\\ HW2 - CNS Sapienza}
\author{Davide Spallaccini - 1642557}
\date{3 of October 2017}
\maketitle


\section{Introduction}
When exchanging messages in a secure way, confidentiality is one of the main concerns of communicating entities. But other important properties of a secure communication are to be taken into account. Message Authentication Codes, that are the subject of this report, are used to provide integrity and at the same time authentication of electronic messages. In order to achieve these properties MAC employ different basic cryptographic well known algorithms, and usually the strength of a MAC is tightly dependent upon the strength of the used algorithms. In particular we will see the inner working of CMAC, a MAC based on block ciphers, and we will compare it to HMAC, which is instead based on cryptographically secure hash functions.

\section{CMAC main features}


\section{CMAC/HMAC comparison}


\section{Conclusions}



\end{document}