\documentclass[11pt]{article}

\begin{document}

\title{CMAC: Cipher-based MAC\\ HW2 - CNS Sapienza}
\author{Davide Spallaccini - 1642557}
\date{3 of October 2017}
\maketitle


\section{Introduction}
When exchanging messages in a secure way, confidentiality is one of the main concerns of communicating entities. But other important properties of a secure communication are to be taken into account. Message Authentication Codes, that are the subject of this report, are used to provide integrity and at the same time authentication of electronic messages. In order to achieve these properties MAC employ different basic cryptographic well known algorithms, and usually the strength of a MAC is tightly dependent upon the strength of the used algorithms. In particular we will see the inner working of CMAC, a MAC based on block ciphers, and we will compare it to HMAC, which is instead based on cryptographically secure hash functions.

\section{CMAC main features}
\paragraph*{Background}
CMAC is known as OMAC which stands for One-Key CBC MAC, and in fact it is a simple variant of CBC MAC (which in turn stands for Cipher Block Chaining MAC). CBC MAC is a technique based on the employment of a block cipher with chaining. The message is originally divided in blocks and encrypted with the well known CBC technique with an initialisation vector of all zeros, the encryption of the last block is chosen as the authentication tag to be send together with the message. The property that is used is that a small change in a bit of the plaintext is propagated in the subsequent blocks until the last one making the tag completely different from the original one, which is the desired characteristic. As most of the MAC variants this requires that communicating parties share a secret key. One of the main problems that CBC MAC implementation have to face is the fact that a naive application of the technique is secure only if the same key is used only for fixed-size messages; for variable length messages some further operations must be carried out. This is why the OMAC1/CMAC variation has been proposed.

\paragraph*{CMAC algorithm}
The basic ingredients of the CMAC technique are an encryption algorithm \textit{E} and a tag length \textit{t}. The encryption algorithm can be any symmetric encryption standard like AES, Triple-DES etc.
The procedure is divided in two parts, in the first part there is a \textit{preprocessing} phase where the message is not needed, in this phase you have to use your secret key to encrypt an all-zeros vector and you have to manipulate the bits according to a certain logic based on bit shifts and xor with a constant depending on the value of the most significant bit obtained with the encryption. This phase is executed twice obtaining two bit strings as result of length equal to the size of a block.
The second phase is the tag generation. Each block of the message from 0 to \textit{m-1} is the encryption of the xor between the block plaintext and the result of the encryption of the previous block (except for the first block). The last block instead is treated differently depending on the length of the block: if the block size is equal to n the block is xor-ed with the first bit string from the preprocessing otherwise it is padded with 10000... until reaching dimension n and the second bit string is used instead. The result is encrypted and is truncated to length \textit{t} giving the tag as the final result. The message is sent together with the tag and the authenticity verification consists of generating the tag from the received message with exactly the same procedure using your own private key and compare it with the received tag, if the comparison is successful the message is authentic and intact, otherwise not.

As pointed out before the security of the technique depends on the strength of the used encryption algorithm, with relative key length.


\section{CMAC/HMAC comparison}


\section{Conclusions}



\end{document}